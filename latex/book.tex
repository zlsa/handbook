\documentclass[oneside,a5paper]{book}
\usepackage[margin={15mm,20mm},footskip=12mm]{geometry}

% For colors
\usepackage{xcolor}

\usepackage{textcase}
\usepackage{titlesec}
\usepackage[xetex]{hyperref}

\usepackage{scrextend}

% Set up PT Serif and Montserrat
\usepackage{xltxtra}
\usepackage{fontspec}

\setmainfont{PT Serif}

\newfontfamily\headingfont[]{Roboto Slab}
\newfontfamily\monofont[]{Roboto Mono}

\definecolor{blue}{HTML}{648DDD}
\definecolor{dark-gray}{gray}{0.2}
\definecolor{light-gray}{gray}{0.35}
\definecolor{white-gray}{gray}{0.5}

% Set up the proper fonts

\titleformat{\chapter}[hang]{}{\color{blue}\huge\bf\headingfont\MakeUppercase\thechapter. }{5pt}{\color{blue}\huge\bf\headingfont\MakeUppercase}[]
\titleformat{\section}[hang]{}{\color{dark-gray}\Large\bf\headingfont\MakeUppercase\thesection. }{5pt}{\color{dark-gray}\Large\bf\headingfont\MakeUppercase}[]
\titleformat{\subsection}[hang]{}{\color{light-gray}\large\bf\headingfont\MakeUppercase\thesubsection. }{4pt}{\color{light-gray}\large\bf\headingfont\MakeUppercase}[]
\titleformat{\subsubsection}[hang]{}{\color{white-gray}\large\bf\headingfont\MakeUppercase\thesubsubsection. }{3pt}{\color{white-gray}\large\bf\headingfont\MakeUppercase}[]

% Set up proper vertical spacing

\titlespacing*{\chapter}{0pt}{-5ex}{0.5ex plus .5ex}
\titlespacing*{\section}{0pt}{2.5ex plus 1ex minus .2ex}{0.6ex plus .2ex}
\titlespacing*{\subsection}{0pt}{1.5ex plus 1ex minus .2ex}{0.5ex plus .2ex}
\titlespacing*{\subsubsection}{0pt}{1.0ex plus 1ex minus .2ex}{0.5ex plus .2ex}

% Disable new page on chapters

\usepackage{etoolbox}

\patchcmd{\chapter}{\thispagestyle{plain}}{\thispagestyle{fancy}}{}{}

\makeatletter
\patchcmd{\chapter}{\if@openright\clearpage\else\clearpage\fi}{}{}{}
\makeatother

% Set line spacing to a larger value

\linespread{1.25}

% Change paragraph separation method

\setlength{\parskip}{4pt}%

% Prepare for the full-bleed cover

\usepackage{wallpaper}
\usepackage[absolute,overlay]{textpos}

\newcommand{\cover}[1]{
  \pagenumbering{gobble}
  
  \ThisCenterWallPaper{1}{#1}

  \begin{textblock}{20}[0, 1](0.8, 15.675)
    {\color{light-gray}\bf\fontsize{8pt}{10pt}\monofont\MakeUppercase\today}
  \end{textblock}

  \begin{textblock}{20}[1, 1](14.675, 15.675)
    {\hspace*{\fill}\color{light-gray}\bf\fontsize{8pt}{10pt}\monofont\projectlink}
  \end{textblock}

  \null\newpage
  
  \pagenumbering{arabic}
}

% Prepare for the illustrations

\usepackage{graphicx}
\usepackage{wrapfig}

% Set up page numbers

\usepackage{fancyhdr}

\pagestyle{fancy}
\fancyhf{}
\lhead{}
\rhead{}
\fancyfoot[C]{\textbf{\large\thepage}}

% Remove the silly line

\renewcommand{\headrulewidth}{0pt}

% Remove enumeration spacing

\usepackage{enumitem}
\setlist{nolistsep}

%%%%%%%%%%%%%%%%%%%%
% CUSTOM MACROS
%%%%%%%%%%%%%%%%%%%%

\newcommand{\link}[2]{\monofont\color{blue}\href{#1}{\textbf{#2}}}

% Define social links

\newcommand{\reddit}[1]{{\link{https://reddit.com/u/#1}{u/#1}}}
\newcommand{\twitter}[1]{{\link{https://twitter.com/#1}{@#1}}}
\newcommand{\email}[1]{{\link{mailto:#1}{#1}}}

% Define the \dv command

\newcommand{\dv}{$\Delta$\hspace{-0.2ex}V}

% Define the illustration command

\newcommand{\widthillustration}[2]{{\centering\includegraphics[width=#1]{../illustrations/#2/#2.png}}}
\newcommand{\illustration}[1]{\widthillustration{\linewidth}{#1}}

% Define vspace after chapters

\newcommand{\afterchapter}{\vspace{1em}}

% Project link macro.

\newcommand{\projectlink}{\href{https://zlsa.github.io/handbook/}{\MakeTextUppercase{zlsa.github.io/handbook}}}

% And here we go.

\title{Orbital Vehicle Operations Handbook}
\author{Jon Ross}

\begin{document}

\begin{titlepage}
\cover{../cover/cover-alpha.pdf}
\end{titlepage}

\makeatletter
\renewcommand\tableofcontents{%
  \@starttoc{toc}%
}
\makeatother

\tableofcontents

\chapter{Introduction}

\afterchapter

Welcome to the Orbital Vehicle Operations Handbook! The Handbook will
guide you through the basics of spacecraft operation and basic orbital
mechanics, in the context of spacecraft operation and flight.

This Handbook covers spacecraft components, orbital mechanics,
rendezvous and docking, aerocapture and reentry, and more. When you're
finished reading this Handbook, you will know how spacecraft
rendezvous with a space station, how capsules control their reentry,
the entire launch process, and more.

I hope this Handbook helps you understand orbital mechanics and better
appreciate the hundreds of thousands of hours that went into
designing, building, and flying any spacecraft.

\vspace{1em}

\begin{itemize}
\item[] \textbf{Jon Ross}
  \vspace{0.5em}
\item[] \reddit{zlsa}
\item[] \twitter{zlsadesign}
\item[] \email{jonross.zlsa@gmail.com}
\end{itemize}

\chapter{Spacecraft Components}

\section{Service Module}

Most crewed spacecraft feature a service module. Much of the equipment
used in space does not need to be safely returned to Earth, such as
solar panels and orbital maneuvering engines. These are usually built
into the service module instead; after the deorbit burn, the service
module is jettisoned before reentry.

\illustration{service-module}

\section{Pressure Vessel}

Every crewed spacecraft needs a pressurized compartment for its
crew. The crew compartment must be pressurized to withstand the vacuum
of space, and it must provide environmental controls and life support
systems (ECLSS) for the long stay in space. The crew compartment
usually contains custom-made seats, for the astronauts to sit in
during launch and reentry, in addition to on-orbit sleeping when not
docked to a space station. The crew compartment is not necessarily the
same as the spacecraft’s outer mold line (the ``shape'' of the vehicle).

\illustration{pressure-vessel}

\subsection{ECLSS}

Environmental controls and life support systems (ECLSS) are a required
component of any crewed vehicle. Without ECLSS, exhaled carbon dioxide
will displace the oxygen, and the crew compartment will overheat when
the capsule is exposed to the sun. ECLSS is a very complicated
mission-critical system, and redundancy is built-in so that an ECLSS
subcomponent failure does not necessitate a mission cancellation.

\section{Power}

Every spacecraft needs power; some spacecraft use fuel cells and some
use batteries. Most battery-powered spacecraft also feature some form
of recharging the batteries while the spacecraft is in space; these
usually take the form of solar panel arrays.

Power is a critical system on any spacecraft since it powers many
other critical systems, such as ECLSS. Because astronauts cannot
simply perform an EVA to debug the power system, it must be very
reliable and foolproof.

\subsection{Solar Arrays}

Solar arrays are used to recharge the onboard batteries. Solar arrays
are almost always installed on the service module, as they’re not
needed during reentry and landing.

Usually, the solar arrays unfold once the spacecraft has reached
orbit; however, some spacecraft feature fixed solar panels that are
mounted to the service module. Fixed panels, unlike deployable solar
arrays, cannot fail to deploy and usually weigh less; however, the
maximum amount of power they generate is limited by the size of the
service module.

\section{Control Systems}

All spacecraft need to maneuver once in orbit, if only to deorbit at
the end of the mission. Most spacecraft also need to keep themselves
pointing in one direction to keep the solar panels facing towards the
sun and to keep the thermal radiator facing away from the sun.

There are a few methods of controlling attitude on a spacecraft, but
by far the most common is a reaction control system (RCS). A reaction
control system is simply a network of small thrusters, that can
produce small amounts of thrust quickly and accurately. With enough
RCS thrusters pointed in different directions, the spacecraft can
orient itself in any direction.

RCS thrusters are usually placed on the capsule or spaceplane, but
some spacecraft feature RCS thrusters on the service module in
addition to the capsule or spaceplane RCS thrusters. RCS propellant is
sometimes stored in the service module as well.

Some spacecraft also feature a dedicated orbital engine, usually
pointed directly backwards. This engine is usually more powerful and
efficient than RCS thrusters; these engines are used when performing
long orbital maneuvering burns, such as the deorbit burn.

Space stations tend to be delicate; typically, visiting spacecraft
must not use their RCS thrusters when in close proximity to the
station. This is to prevent damage of delicate space station
components, such as solar arrays and wiring.

\chapter{Orbital Mechanics}

\section{Delta-V}

Delta-v, dV, or \dv\ (short for delta velocity), is simply the
difference in velocity between two states. For example, imagine your
spacecraft is floating in space, with no other bodies to affect it; to
quantifiably measure \dv\ , a part of the spacecraft is separated from
the rest of the spacecraft. Because of Newton’s first law, it will
simply coast alongside the spacecraft; the current \dv\ between the
spacecraft and the part that was left behind is 0 meters per second.

Now imagine a rocket engine on the spacecraft is ignited. The
spacecraft will start to fly away from the part that was left
behind. When the engine is shut down, the \dv\ of the burn is simply the
difference in velocity between the piece of your spacecraft (which is
equivalent to the velocity of the spacecraft before the burn) and your
spacecraft after the burn. \dv\ is almost always measured in meters per
second.

\illustration{delta-v}

\dv\ does not measure distance; it measures only a difference in
velocity. The energy required for maneuvers and engine burns is
measured in \dv\ , providing a vehicle-agnostic measurement of the amount
of energy required.

\section{Gravity}

While a spacecraft is in orbit, gravity is still acting upon it even
though it may appear to be ``weightless''. This is because the
spacecraft, along with its crew and cargo, are all affected only by
gravity, without any friction; since the external gravitational
influence is identical for both the spacecraft and its contents;
hence, the relative velocity between the spacecraft and its contents
is zero, creating the illusion of zero-G.

\section{Basic Concepts}

\subsection{Orbits}

The path that a body traces through space while under the
gravitational influence of another body is called an orbit. It can be
tricky to understand at first, since orbits, by necessity, do not work
on the surface of a planet, so for most people, there is nothing to
base orbital knowledge on.

A simple way to visualize an orbit is called Newton’s Cannonball. The
thought experiment involves a small planet with a large mountain; on
the very top of this mountain is a cannon, pointed horizontally (not
at the horizon; since the cannon is on a mountain, the horizon would
be below horizontal). If you were to just drop a cannonball from the
mountain, it would immediately fall down until it hit the mountain; if
you threw it as hard as you could it would fly forwards a little, but
it would still hit the mountain.

\begin{center}
  \widthillustration{0.5\linewidth}{newton-cannonball}
\end{center}

\vspace{-2em}

Now, if you put the cannonball into the cannon and fired the cannon,
the cannonball would travel very far indeed, but it would eventually
hit the ground, many miles downrange from the mountain. If you had a
light enough cannonball and a small enough planet, you would be able
to fire a cannonball so fast that it would travel forwards as it fell
``around'' the planet. As the cannonball falls towards the planet, its
momentum carries it forwards. In the absence of an atmosphere, there
would be no air friction; so the cannonball would stay in orbit
forever. (However, as a side note, the cannonball would circle around
the planet and strike the cannon from the other side. Since no forces
have acted on the cannonball other than the planet, its path will not
change during its orbit, so it will arrive exactly where it started
from, but with a lot of velocity.)

Technically, when you throw a ball into the air, the ball is in a
highly elliptical orbit with an apoapsis within the atmosphere; it has
a definite periapsis and apoapsis. However, the Earth is so large that
unless you’re throwing things very fast, there will only be a tiny
difference between an elliptical orbit and a simplified parabolic
trajectory.

\subsection{Two-body Systems}

In an ideal two-body system, both of the bodies will orbit around
their common barycenter. This is because both bodies are affecting
each other; no matter how heavy one is and how light the other is, the
bodies will have a barycenter. As one body gets heavier and the other
gets lighter, the barycenter will move towards the heavier body;
however, unless the lighter body has zero mass, the heavier body will
always be orbiting the shared barycenter.

\illustration{barycenter}

Despite this, in an ideal two-body system featuring a planet (such as
the Earth or Mars) and a spacecraft, it can usually be simulated as a
fixed, nonmoving body and a spacecraft that orbits around it. For
example, in an ideal two-body system with a typical crewed spacecraft
weighing about 8 metric tons orbiting the Earth, the barycenter of the
pair is only one 50,000th of an atom away from the Earth’s center of
mass (CoM). Realistically, the barycenter can be ignored when planning
most spacecraft missions.

\subsection{Kepler's Three Laws}

Kepler’s three laws describe the elliptical motion of a body orbiting
a planet. (Note that these three laws only apply in the case of an
orbit with an eccentricity less than one.)

\begin{enumerate}
\item The orbit of a body around a planet is an ellipse, with the planet
  being at one of the two foci of the ellipse.
\item A line segment connecting the body and the planet sweeps out equal
  areas during equal areas of time.
\item The square of the orbital period of a body is proportional to the
  cube of the semimajor axis of its orbit.
\end{enumerate}

The first law simply describes the basic shape of any orbit; the
second describes the time taken to travel around different parts of
the orbit; and the third states that orbital speed is relative to the
size of the ellipse.

\illustration{kepler-second-law}

\subsection{Multi-body Systems}

The above information is only correct for an idealized two-body
system. In reality, while your spacecraft may only be orbiting a
single planet, that planet is orbiting another body, which may have
dozens of bodies orbiting it, all of which also have other orbiting
bodies.

During Earth operations on-orbit, the strongest gravitational
influences are the Earth, the Moon, the Sun, and Jupiter. The Earth
and the Moon are high on the list for obvious reasons, as is the
sun. Jupiter is also on the list because it’s an exceptionally heavy
planet (over three times heavier than Saturn, the next heaviest
planet). Despite being so far away, it exerts a measurable influence
on satellites orbiting the Earth, and it must be taken into account
during the mission planning phase.

\illustration{perturbations}

When a spacecraft is affected by forces not taken into account in an
ideal two-body system, the orbital trajectory includes
perturbations. These can be caused by a non-spherical planet (such as
the earth), an uneven mass distribution of the planet, or other
planets. Perturbations manifest as slight drifts to the orbit over
time; the spacecraft may need to perform correction burns as a
result. The Earth itself isn’t perfectly spherical; its gravitational
force varies depending on the spacecraft’s inclination. Of other
planets, the Earth’s moon is very large relative to the Earth for a
moon; its effects on orbiting spacecraft is marked, and special care
must be taken during mission planning for lunar perturbations.

Mathematical equations have been developed to quantify the influence
the moon has on Earth-based spacecraft orbits; together, they are
called lunar theory. Taking into account lunar theory is an important
part of planning Earth-based missions since the moon strongly affects
any satellite or spacecraft orbiting Earth due to its high mass and
relatively close proximity to the Earth.

\subsection{Eccentricity}

So far, we have only investigated perfectly circular orbits. In
reality, no orbit is perfectly circular; such an ideal orbit has an
eccentricity of zero. The more elliptical an orbit is, the higher its
eccentricity; an orbit with an eccentricity of one is a parabolic
escape trajectory; and an orbit with an eccentricity greater than one
is a hyperbolic trajectory.

\illustration{eccentricity}

When a body is orbiting around its parent with a parabolic escape
trajectory, its speed is the absolute minimum required to escape the
gravitational influence of its parent; if its speed were any lower,
the eccentricity would be less than one, and it would remain in orbit
instead of being flung out.

A hyperbolic trajectory is simply a parabolic escape trajectory, but
with a higher velocity. In reality, any spacecraft traveling between
two bodies will use a hyperbolic trajectory, both when departing the
first body and when arriving at the second.

\subsection{Inclination}

Unlike the orbits explained in these diagrams, reality is not
two-dimensional. Spacecraft must deal with a third dimension:
inclination.

When launching from an ideal equatorial launch site, the inclination
of the resulting orbit can be as low as zero degrees. The ground track
of a zero-inclination orbit would follow along the equator; if you
mounted a camera onboard the spacecraft and aimed it at the Earth, the
equator would always cross through the center of the frame.

\illustration{inclination}

However, no launch sites currently in operation are situated on the
equator. Cape Canaveral is at 28 degrees north of the equator, while
Russia’s Baikonur Cosmodrome is nearly 46 degrees north of the
equator. When a spacecraft is launched into orbit, the minimum
inclination is the same as that of its launch site. This is a large
part of the decision to build the ISS at 51.6 degrees inclination,
since Russia’s launch site is located at 46 degrees north; if the
inclination of the ISS was lower than 46 degrees, Russia would not be
able to easily launch their spacecraft to the ISS.

There’s another factor that must be taken into consideration in
conjunction with orbital inclination, and that’s the velocity of the
planet’s surface at the launch site. Since the Earth spins along its
axis once every 23 hours and 56 minutes (one sidereal day), some
velocity can be taken from the Earth’s rotation when launching
spacecraft. At the equator, the Earth’s surface velocity (relative to
the Sun and Earth’s center of mass) is 464 meters per second
(m/s). The further the launch site is from the equator, the less
surface velocity is present. To launch into orbit from a high
inclination launch site requires more energy from the launch vehicle
as compared to an equatorial launch site.

\illustration{latitude-velocity}

This is why most launch sites are located as close as possible to the
Earth’s equator: the closer the launch site is to the equator, the
less fuel a launch vehicle will need to reach Earth orbit (and
beyond).

\subsection{Periapsis and Apoapsis}

The periapsis of an orbit is the point at which the spacecraft and
body are the closest to each other; the apoapsis is simply the
furthest point. Any orbit with an eccentricity greater than one will
have an apoapsis and a periapsis. The periapsis and apoapsis are
typically measured from the spacecraft to the surface of the body, to
simplify matters for the astronauts; however, mathematically, the
periapsis and apoapsis are measured relative to the barycenter.

\illustration{periapsis-apoapsis}

Since periapsis and apoapsis describe locations relative to the orbit,
not the parent body, any changes to the orbit will also change the
periapsis and apoapsis.

\subsubsection{Semi-major Axis}

The semimajor axis of an orbit is the distance from the periapsis to
the apoapsis, divided by two. Despite appearing to be derived from the
periapsis and apoapsis, the semimajor axis is the root from which
those are derived; since an orbit with an eccentricity less than one
is an ellipse with the parent body at one of the foci, the semimajor
axis is generally more useful for orbital calculations than the
periapsis and apoapsis are.

\illustration{semimajor-axis}

\subsubsection{Orbital Period}

The period of an orbit is the amount of time it takes to complete one
full orbit. It can be calculated from the semimajor axis and the mass
of the parent body. Due to Newton’s laws, a higher orbit (with a
larger semimajor axis) will have a longer period than that of a lower
orbit. To adjust the orbital period, the semimajor axis must change;
this is done by changing the apoapsis and the periapsis by burning
prograde or retrograde.

\subsection{Ascending and Descending nodes}

Every object orbiting a body with an inclination greater than zero
will have both an ascending node and a descending node in its
orbit. Like the periapsis and apoapsis, the ascending and descending
nodes are located relative to the orbit and the parent body, so any
changes to the orbit will also change the ascending and descending
node.

Ascending and descending nodes only work if the parent body has an
equator. The imaginary plane extending out in all directions from the
equator of the parent body is called the plane of reference; it
applies to any satellites orbiting the body.

The ascending node is the point where the orbit crosses the plane of
reference, travelling south to north. The descending node simply
delineates the opposite side of the orbit, where it crosses the plane
of reference, travelling north to south.

\subsubsection{Argument of Periapsis}

The argument of periapsis is the angle, measured starting from the
ascending node and travelling northwards, to the periapsis. The
argument of periapsis orients the orbit’s semimajor axis within its
orbital plane.

\subsubsection{Longitude of the Ascending Node}

The longitude of the ascending node is the longitude, relative to the
parent body’s center of mass, is the longitude directly underneath the
ascending node. It is measured relative to the parent body’s center of
mass, not its surface; this means that the longitude of the ascending
node will not change as the parent body spins on its axis.

The longitude of the ascending node is used to orient the orbit
relative to the parent body.

\subsection{Mean Anomaly at Epoch}

The mean anomaly at epoch describes the virtual angular position of
the satellite at a predetermined time (the epoch). The mean anomaly
does not describe a physical angle, except in the case of a perfectly
circular orbit; because of Kepler’s third law, any orbit with an
eccentricity greater than zero will have a non-uniform orbital
velocity. Mean anomaly will therefore ``drift'' during the orbit; so
it must first be converted into true anomaly. True anomaly takes into
account the eccentricity of the orbit, and so provides the direct
physical angle between the periapsis and the orbiting object at a
given point in time.

\subsection{Keplerian Elements}

When the values for eccentricity, inclination, semimajor axis, the
argument of periapsis, the longitude of the ascending node, and mean
anomaly at epoch are combined, the exact position of the spacecraft in
space can be calculated at any time. Together, these six values are
called the orbit’s Keplerian elements. They only apply to an ideal
two-body system, with no external gravitational influences. Since
external gravitational influences affect the orbit over time,
Keplerian elements cannot be accurately used for anything beyond an
ideal two-body system.

\section{Orbital Maneuvering}

\afterchapter

Maneuvering in orbit is a critical component of any spacecraft
operation, and is required for orbital rendezvous and docking as well
as deorbit operations. It is also essential during any mission that
flies beyond low earth orbit.

Orbital mechanics make maneuvering difficult and counter-intuitive.

All orbital maneuvers simply adjust your orbit, albeit in different
directions. There are a few main categories of orbital maneuvering:

\vspace{2em}

\begin{enumerate}
\item Periapsis and apoapsis adjustments (changing your altitude)
\item Orbit phasing (adjusting the duration, or period, of your orbit)
\item Plane changes (also known as inclination changes)
\end{enumerate}

\subsection{Impulsive vs. Finite Maneuvers}

Any maneuver can be ideally simulated as an impulsive maneuver, a
maneuver which is completed instantly. In reality, an impulsive
maneuver is impossible as it would require infinite thrust; however,
it’s a good approximation during mission planning. The analogous
non-instant maneuver is called a finite maneuver, as it takes a finite
amount of time to complete.

There are methods to convert an impulsive maneuver into a nearly
equivalent finite maneuver; thus, most of the mission planning can be
accomplished with impulse maneuvers, to be converted into finite
maneuvers after verifying the intermediate orbits.

The following sections will assume impulse maneuvers for
simplicity. In reality, these maneuvers will take a finite amount of
time, so the start of the maneuver must be shifted backwards in time
so that the midpoint of the maneuver is at the proper location.

\subsection{Orientation}

As you might guess, orientation is critical during orbital maneuvers,
because most spacecraft have an optimum direction to thrust in. Some
spacecraft have engines that are specifically designed for orbital
maneuvers, but they only point in one direction. For these reasons,
spacecraft need to orient themselves while performing orbital
maneuvers.

Most maneuvers involve orienting the ship either prograde (in the
direction of movement) or retrograde (the exact opposite of
prograde). Inclination changes, however, require orientations that
point transversely relative to the orbital plane (i.e. ``north'' or
``south'', but relative to the orbital plane). These are called normal
(for north-facing) and anti-normal (for south-facing).

Very rarely used are radial orientations; a radial-in orientation is
perpendicular to the orbital trajectory, pointing in the general
direction of the planet; a radial-out orientation is the opposite.

\subsection{Periapsis and Apoapsis Adjustments}

To adjust the periapsis or apoapsis, the spacecraft must be at the
opposite point in the orbit. For example, to increase the apoapsis,
the spacecraft must be at the periapsis, then burn prograde until the
apoapsis reaches the target value. The reverse is true to lower the
apoapsis.

\illustration{raise-apoapsis}

Since the orbital period depends on the length of the semimajor axis,
with a larger orbit having a longer orbital period, adjusting the
periapsis and apoapsis will by necessity also adjust the orbital
period.

Often, a spacecraft needs to adjust both its periapsis and apoapsis by
the same amount. Usually, the most efficient method is a Hohmann
transfer. To perform a Hohmann transfer, the spacecraft first performs
an orbit adjustment maneuver to change the altitude of the other side
of its orbit; then, when it reaches the new altitude at the other side
of its orbit, it performs another orbit adjustment burn to bring the
first point to the proper altitude as well.

\subsection{Orbit Phasing}

Orbit phasing refers to the timing of a spacecraft within its
orbit. This is usually required during rendezvous; even if the
semimajor axes, orbital planes, and eccentricities of the two
spacecraft are the same, the spacecraft may be a half-orbit
apart.

Orbit phasing needs to be performed on every mission that includes a
rendezvous. However, with the proper planning, it can be performed
alongside the spacecraft’s altitude adjustment maneuvers.

\subsection{Plane Changes}

A plane change maneuver adjusts the inclination of the spacecraft’s
orbit. To do so, it must burn in the normal or anti-normal
directions. Plane change maneuvers are the most expensive maneuver, \dv\ 
wise. To adjust the inclination of a circular orbit by 90 degrees
requires nearly as much \dv\ as it took to reach orbit in the first
place. Plane changes are only performed when they are absolutely
necessary, such as for satellites that must be in geostationary
equatorial orbit (GEO). Such satellites must be on the equator, and
since no launch site is on the equator, they must perform a plane
change once in orbit.

Since all orbital maneuvering \dv\ is relative to your existing \dv\ ,
plane changes are more efficient the more eccentric your orbit is,
provided the plane change maneuver is performed at the apoapsis. It’s
often more efficient to dramatically increase the apoapsis, perform a
plane change, then lower the apoapsis again instead of performing a
plane change in-place. Some launch providers will launch GEO
satellites into a super-synchronous orbit, with an apoapsis much
higher than necessary, to reduce the \dv\ requirements of the satellite
itself.

\subsection{Oberth Effect}

The Oberth effect describes the counterintuitive fact that a prograde
burn at periapsis imparts more energy to the spacecraft than one
performed at the apoapsis. This is because the kinetic energy in an
object is the square of its velocity; therefore, adding a fixed amount
of velocity while the spacecraft is moving rapidly will add more
kinetic energy than if the velocity was added while the spacecraft was
moving slowly.

\section{Orbital Rendezvous}

\afterchapter

One of the most difficult tasks is rendezvous and docking with another
spacecraft. It requires a deep understanding of orbital maneuvering,
the capabilities of your spacecraft, and the ability to precisely
target another spacecraft in orbit.

To understand orbital rendezvous, it’s simpler to first work backwards
from a spacecraft that’s already docked to a space station. Their
orbits are completely identical, and their speeds are matched up
perfectly.

First, the spacecraft must undock from the space station. This is
usually performed by sealing the hatch from both sides, unlocking the
docking port, then using electromechanical pushers to mechanically
separate the two. After some coast time, the spacecraft is far enough
away from the space station to use its RCS thrusters.

At this point, the spacecraft will begin a series of Hohmann
transfers, reducing its altitude (and increasing its distance to the
space station at the same time). This is done to minimize RCS thruster
usage while the spacecraft is still near the space station. From an
observer on the space station, the spacecraft will appear to fall
towards the planet and start moving retrograde.

Now, the spacecraft is in a separate orbit, lower and faster than that
of the space station. The spacecraft has just performed the opposite
of a rendezvous; the only difference is that a departure is typically
faster than a rendezvous, as there’s less chance of approaching the
space station too closely.

The reverse of spacecraft departure is spacecraft rendezvous. Now, the
spacecraft is in the exact same low, fast orbit that is needed to
begin orbital rendezvous; with the space station in a higher, slower
orbit. The orbital planes are parallel; if not, a plane change
maneuver will have to be made.

The spacecraft now performs a series of Hohmann transfers, to raise
its orbit to that of the space station. The timing of these transfers
is critical, since the orbital phasing is being adjusted at the same
time. During the maneuvering, the spacecraft is communicating with
both the space station and ground controllers, who watch the position
to make sure the spacecraft will not pass too close to the space
station.

As the spacecraft approaches the space station from below, it performs
several small burns to match orbits nearly perfectly, then maneuvers
itself around the space station to a docking port; despite the close
proximity of the spacecraft and the space station, orbital mechanics
still applies, and care must be taken that the spacecraft remains in
the same position relative to the space station.

The only thing that remains is to dock with the space station; this is
usually fully automated, like rendezvous; during the docking
procedure, the spacecraft will use its onboard docking camera to track
visual targets on the space station’s docking port.

When the spacecraft contacts the docking port, petal-shaped flaps
align the two docking ports; hooks extend to lock the two together,
and connectors for air and electricity are connected and opened. Soon
afterwards, the hatches on both sides are opened and crewmembers may
board the space station.

\chapter{Mission Operations}

\afterchapter

In reality, a crewed mission is much more complex than the orbital
maneuvers described above. In fact, with a mission that may only last
a few days, mission planners on the ground have been preparing the
mission for several years.

\section{Launch Operations}

\subsection{Prelaunch Operations}

Months before the launch of your spacecraft, mission planners will begin to plan out your mission. If your mission is to a space station such as the International Space Station, the mission planners will need to secure approval for a visiting spacecraft as well as preparing for the rendezvous procedure.

Your mission planners will formulate a mission plan. This includes:

\begin{itemize}
\item Cargo manifest and center of mass calculations
\item Astronaut scheduling
\item Launch profile and orbital trajectory
\item Space station approach, rendezvous, and docking
\item Space station undocking and departure
\item Deorbit planning and entry trajectory
\item Recovery vessel positioning
\end{itemize}

While your mission is being prepared by the mission planners, you and
your crew will train for spacecraft operations and zero-g movement, in
addition to whatever specialized training your mission may
require. Crew training is considered one of the more strenuous and
difficult parts of being an astronaut, but just remember: what you
learn in crew training will possibly save your life in the future.

\subsection{Launch Operations}

The spacecraft will arrive at the launch site integration hangar a few
weeks before launch. During this time, it is loaded with cargo, with
the exception of late-load cargo. Checks are done on the spacecraft,
and it’s integrated to the launch vehicle and readied for rollout and
erection.

The launch vehicle is usually rolled out to the launchpad a few days
in advance of T-0. Different launch providers have different launch
vehicle designs, and hence have different rollout and erection
schedules and guidelines.

You and your crew will board the spacecraft only a few hours before
launch, before fueling has occurred. Typically, you will board the
spacecraft via the crew arm, which is swung away after ground crew
have strapped you into your seats and closed the spacecraft ground
door.

After the launchpad is cleared of all personnel and the command is
given, launch vehicle fueling will begin. Once again, precise timing
depends on the launch provider’s choice of vehicle.

If your mission includes rendezvous with a space station, the launch
window is very short or instantaneous. If the launch occurs too far
from the optimal launch time, the spacecraft will need to perform an
expensive on-orbit inclination change. Most spacecraft do not have
enough onboard delta-v to perform this maneuver themselves.

Fueling is usually completed a few minutes to an hour before the
planned liftoff time. The crew access arm retracts, and the launch
vehicle and spacecraft switch to internal power (instead of power
provided from the launch pad) in preparation for flight.

A few minutes before liftoff, the fueling valves are closed, and any
excess gases in the tanks are vented until seconds before engine
ignition. This is only an issue with cryogenic propellants, such as
liquid oxygen or liquid methane; kerosene, with a boiling temperature
well above the boiling point of water, does not boil off while in the
tanks.

A few seconds before liftoff, the engine ignition command is
given. Engine startup is a long and complicated process, and
auto-aborts during the ignition sequence are not rare. After the
vehicle has determined that all of the engines are operating properly
and at the right thrust levels, the command is given to release the
hold-downs.

As soon as the hold-downs are released, the launch vehicle will begin
to ascend. A few seconds after liftoff, the launch vehicle will have
cleared the tower; soon afterwards, the vehicle will perform a pitch
kick to begin the gravity turn.

As the propellants are burned in the engines to produce thrust, the
launch vehicle will lose mass; however, since the thrust level of the
engines does not decrease, the thrust-to-weight ratio (TWR) will
increase during first-stage flight. To compensate for increased
acceleration, launch vehicles need to reduce thrust. There are two
primary ways to reduce thrust; the method depends on the
throttleability of the launch vehicle’s chosen rocket engine.

On launch vehicles whose engines cannot throttle (or if the engines do
not have the necessary throttle range), one or more engines are shut
down to reduce total thrust. This method only works if you have three
or more engines, and if they are arranged to allow for shutdowns
without adversely affecting the thrust vector.

The other, more common method is to throttle all of the engines down
equally. This is the preferred method for many reasons: first,
throttling engines is much smoother than shutting them down; second,
engine shutdown is nearly as complex as engine ignition; and third,
shutting down engines changes the thrust distribution. All
currently-flying human-rated launch vehicles throttle their engines
during the ascent.

During the atmospheric portion of ascent, the launch vehicle must fly
in the direction it’s travelling; even slight deviations will cause a
rapid loss of control due to aerodynamic forces and necessitate a
launch abort.

As the vehicle ascends, there will be a point of maximum dynamic
pressure (MaxQ) on the vehicle; this is the point of highest
aerodynamic stress on the vehicle and is one of critical points during
a launch. In most launch vehicles, MaxQ occurs shortly after the
launch vehicle goes supersonic.

When the target velocity and altitude are reached, the first stage
engines shut down; this is called main engine cutoff (MECO). Shortly
afterwards, the first stage is separated and the second stage engine
is ignited.

Depending on how many stages the launch vehicle has, this cycle
continues until the spacecraft has reached its desired orbit (or in
the case of a mission to a space station, a preliminary parking
orbit). At this point, the engine on the final stage is shut down and
the spacecraft is separated from the launch vehicle.

\subsection{Postlaunch Operations}

The primary concern after reaching orbit is power. Most spacecraft
feature unfolding solar arrays, but some spacecraft, such as the
SpaceX Crew Dragon, feature integrated solar panels that do not
require any action to provide power. On the other hand, the Boeing
Starliner does not feature any type of power generation; as such, the
Starliner is only capable of visiting space stations that can provide
power to visiting vehicles.

Once the spacecraft is in the correct attitude and is generating
power, less important systems can be checked for proper functionality.

\chapter{Planetary Landing}

\afterchapter

Successfully landing on a planet requires a set of steps to be
performed with precise timing. First, the spacecraft’s periapsis must
be low enough to capture the spacecraft as it descends to the
atmosphere; second, the spacecraft must control itself during reentry
to target a landing point; and third, the spacecraft must provide some
form of deceleration to allow for a smooth touchdown.

\section{Deorbit}

Deorbiting a spacecraft is relatively simple: you just perform a
maneuver at the apoapsis that lowers the periapsis and fine-tunes the
entry trajectory. When the perigee is deep enough in the atmosphere,
deorbit is assured. Once the spacecraft maneuvers are complete, the
service module, if present, is jettisoned, and the spacecraft
reorients itself in preparation for atmospheric entry.

\section{Atmospheric Entry}

After less than a half-orbit, the spacecraft will reach the
atmospheric entry interface and engage active entry guidance. Every
modern crewed spacecraft is capable of guided entry; even
capsule-derived spacecraft can control their trajectory with an offset
center of mass and precisely controlled roll around the vertical
axis. Most spaceplanes perform wide S-turns to decrease peak G-forces.

\illustration{reentry-capsule-control}

During the time of peak heating and G-forces on the spacecraft,
communications will cut out as a consequence of the superheated plasma
surrounding the spacecraft; communications blackout typically lasts
only a few minutes. During this time, the spacecraft experiences high
G-forces, typically in the range of 3-5 Gs (that is, 3-5 times more
force than on Earth); spaceplanes usually experience less G-force as a
virtue of their increased lift-to-drag ratio. Every crewed spacecraft
is designed for these forces, and furthermore, they’re designed to
safely and comfortably secure the astronauts during this time.

\section{Landing}

\subsection{Parachute Splashdown/Airbag Assist}

Most modern capsule-derived spacecraft use parachutes for a splashdown
in the ocean or an airbag or rocket-assisted land landing. This method
is reliable and proven, and is relatively simple compared to other
choices. This method deploys drogue chutes after the capsule
decelerates to subsonic velocities, then uses the drogue chutes to
pull out the main chutes, of which there are usually three or
four. After a few minutes of slow descent under the parachute canopy,
the capsule either splashes down in the ocean or deploys airbags or
rocket motors, to soften a land-based touchdown.

Parachute landings are the tried-and-true technique, but they suffer
from difficult recoveries (in the case of ocean splashdowns) and hard
landings, even when airbag or rocket-assisted (in the case of land
landings). Ocean splashdowns also expose the spacecraft to salty ocean
water only minutes after the spacecraft has been exposed to the
extreme temperatures of atmospheric entry.

\subsection{Propulsive Landing}

Some next-generation spacecraft feature propulsive landing. Instead of
using a parachute to reduce their speed, they use multiple onboard
rocket motors to slow the vehicle down in a precise,
computer-controlled manner. The primary advantage of propulsive
landing is the increased accuracy and no chance of tangled parachutes;
on the other hand, propulsive landing is a relatively new addition to
crewed capsules, and reliability hasn’t been proven yet. In addition,
propulsive landing does not depend on an atmosphere; propulsive
landing works just as well on Mars as it does on Earth.

Propulsively-landed spacecraft cannot be landed manually; humans
simply cannot control the spacecraft quickly or accurately enough for
a manual landing. Spacecraft with propulsive landing will need to have
a parachute backup in the near future until the reliability of
propulsive landing is proven.

\subsection{Spaceplane Runway Landing}

Currently, there are no crewed spaceplane designs; but in the future,
Sierra Nevada Corporation’s DreamChaser vehicle might change
this. Nevertheless, we will provide a description of what to expect
when landing a spaceplane on a runway.

Unlike the far simpler parachute splashdowns in the ocean, a runway
landing requires highly accurate computerized entry guidance, to
ensure the spacecraft is able to reach the target runway. Furthermore,
no spaceplane currently in service features atmospheric engines;
therefore, the flight must be flown unpowered.

As airplanes cannot gain altitude without losing speed, all spaceplane
entry trajectories must place the spaceplane on a trajectory that
overflies the runway; after reentry, as the spaceplane slows down from
hypersonic speeds, it will perform wide S-turns to bleed off speed and
altitude without gaining as much ground as a direct glide would.

When the spaceplane reaches the appropriate altitude and is aligned to
the runway, the spaceplane will autonomously descend along the
glideslope, deploy its landing gear, and rollout along the runway
automatically. Modern spaceplanes are difficult to control manually,
so the pilot will only take over in the case of an anomaly with the
autopilot.

\chapter{Planetary Operations}

\section{Departing from a planet}

To depart from a planet, the spacecraft must be in an orbit with an
eccentricity greater or equal to one; i.e. a hyperbolic orbit. The
maneuver used to raise the apogee beyond the planet’s sphere of
influence is called an escape burn.

Escape burns are more efficient at the periapsis of the spacecraft’s
orbit due to the Oberth effect; to perform an escape burn, the
spacecraft simply raises its apogee until the eccentricity of its
orbit is greater than or equal to one. Escape burns must be precisely
plotted ahead-of-time to ensure the escape trajectory intersects the
destination planet at the right time, angle, and speed.

\section{Arriving at a Planet}

When arriving at a planet, the spacecraft will be on a hyperbolic
trajectory; to enter an orbit, it must perform a capture maneuver
which slows the vehicle down at its periapsis. To do so, it must use
some combination of a retrograde engine burn and aerobraking.

\subsection{Retrograde Engine Burn}

A retrograde engine burn capture is performed by orienting the
spacecraft retrograde, then performing a long engine burn at periapsis
to reduce the spacecraft’s apoapsis. This method does not require a
heatshield and works on planets without atmospheres.

\subsection{Aerobraking}

Aerobraking is a maneuver which uses the atmosphere of a planet to
lower the spacecraft’s apogee without landing. The spacecraft must
have a shield of some sort to dissipate the heat generated during
atmospheric flight, and the planet must have an atmosphere to brake
against.

This method is far superior to an engine burn in terms of mass
penalty, but it imposes strict limits on the spacecraft’s design, as
it must withstand very high G-forces. In addition, the spacecraft
trajectory must be accurately controlled to ensure a precise orbit
after the maneuver. This is because atmospheres vary over time, and
the overall braking force can’t be calculated accurately beforehand.

\newpage

\pagenumbering{gobble}

\topskip0pt

\vspace*{\fill}

\begin{centering}
  {\color{light-gray}\large\bf\headingfont\MakeTextUppercase{Last updated on}} \\
  \vspace{1.5em}
  {\color{blue}\huge\bf\headingfont\MakeUppercase\today} \\
  \vspace{1em}
  {\color{light-gray}\bf\monofont\projectlink} \\
  \vspace{3em}
\end{centering}

\vspace*{\fill}

\end{document}
