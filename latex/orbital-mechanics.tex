
\chapter{Orbital Mechanics}

\section{Delta-V}
\label{sec:dv}

Delta-v, dV, or \dv\ (short for delta velocity), is simply the change
in velocity between two velocities. \dv\ is a difficult concept to wrap
your mind around at first, so here's an example of \dv\ in action.

Imagine a spacecraft that's floating in space, with no other bodies to
affect it; to quantifiably measure \dv, a part of the spacecraft is
separated from the rest of the spacecraft. Because of Newton's first
law, it will simply coast alongside the spacecraft; the current
\dv\ between the spacecraft and the part that was left behind is 0
meters per second.

Now imagine a rocket engine on the spacecraft is ignited. The
spacecraft will start to fly away from the part that was left
behind. When the engine is shut down, the \dv\ of the burn in this
case is simply the difference in velocity between the piece of your
spacecraft (which is equivalent to the velocity of the spacecraft
before the burn) and your spacecraft after the burn.

Of course, in reality, \dv\ isn't measured by dropping off parts of
the spacecraft; instead, it can be calculated from the efficiency of
an engine and the total mass of the spacecraft before and after the
burn.

\dv\ is almost always measured in meters per second.

\illustration{delta-v}

\dv\ does not measure distance; it measures only a difference in
velocity. The energy required for maneuvers and engine burns is
measured in \dv\ , providing a vehicle-agnostic measurement of the amount
of energy required.

\section{Gravity}

While a spacecraft is in orbit, gravity is still acting upon it even
though it may appear to be ``weightless''. This is because the
spacecraft, along with its crew and cargo, are all affected only by
gravity, without any friction; since the external gravitational
influence is identical for both the spacecraft and its contents;
hence, the relative velocity between the spacecraft and its contents
is zero, creating the illusion of zero-G.

\section{Basic Concepts}

\subsection{Orbits}

The path that a body traces through space while under the
gravitational influence of another body is called an orbit. It can be
tricky to understand at first, since orbits, by necessity, do not work
on the surface of a planet, so for most people, there is nothing to
base orbital knowledge on.

A simple way to visualize an orbit is called Newton's Cannonball. The
thought experiment involves a small planet with a large mountain; on
the very top of this mountain is a cannon, pointed horizontally (not
at the horizon; since the cannon is on a mountain, the horizon would
be below horizontal). If you were to just drop a cannonball from the
mountain, it would immediately fall down until it hit the mountain; if
you threw it as hard as you could it would fly forwards a little, but
it would still hit the mountain.

\begin{center}
  \widthillustration{0.5\linewidth}{newton-cannonball}
\end{center}

\vspace{-2em}

Now, if you put the cannonball into the cannon and fired the cannon,
the cannonball would travel very far indeed, but it would eventually
hit the ground, many miles downrange from the mountain. If you had a
light enough cannonball and a small enough planet, you would be able
to fire a cannonball so fast that it would travel forwards as it fell
``around'' the planet. As the cannonball falls towards the planet, its
momentum carries it forwards. In the absence of an atmosphere, there
would be no air friction; so the cannonball would stay in orbit
forever. (However, as a side note, the cannonball would circle around
the planet and strike the cannon from the other side. Since no forces
have acted on the cannonball other than the planet, its path will not
change during its orbit, so it will arrive exactly where it started
from, but with a lot of velocity.)

Technically, when you throw a ball into the air, the ball is in a
highly elliptical orbit with an apoapsis within the atmosphere; it has
a definite periapsis and apoapsis. However, the Earth is so large that
unless you're throwing things very fast, there will only be a tiny
difference between an elliptical orbit and a simplified parabolic
trajectory.

\subsection{Two-body Systems}

In an ideal two-body system, both of the bodies will orbit around
their common barycenter. This is because both bodies are affecting
each other; no matter how heavy one is and how light the other is, the
bodies will have a barycenter. As one body gets heavier and the other
gets lighter, the barycenter will move towards the heavier body;
however, unless the lighter body has zero mass, the heavier body will
always be orbiting the shared barycenter.

\illustration{barycenter}

Despite this, in an ideal two-body system featuring a planet (such as
the Earth or Mars) and a spacecraft, it can usually be simulated as a
fixed, non-moving body and a spacecraft that orbits around it. For
example, in an ideal two-body system with a typical crewed spacecraft
weighing about 8 metric tons orbiting the Earth, the barycenter of the
pair is only one 50,000th of an atom away from the Earth's center of
mass (\acrshort{com}). Realistically, the barycenter can be ignored
when planning most spacecraft missions.

\subsection{Kepler's Three Laws}

Kepler's three laws describe the elliptical motion of a body orbiting
a planet. (Note that these three laws only apply in the case of an
orbit with an eccentricity less than one.)

\begin{enumerate}
\item The orbit of a body around a planet is an ellipse, with the planet
  being at one of the two foci of the ellipse.
\item A line segment connecting the body and the planet sweeps out equal
  areas during equal areas of time.
\item The square of the orbital period of a body is proportional to the
  cube of the semi-major axis of its orbit.
\end{enumerate}

The first law simply describes the basic shape of any orbit; the
second describes the time taken to travel around different parts of
the orbit; and the third states that orbital speed is relative to the
size of the ellipse.

\illustration{kepler-second-law}

\subsection{Multi-body Systems}

The above information is only correct for an idealized two-body
system. In reality, while your spacecraft may only be orbiting a
single planet, that planet is orbiting another body, which may have
dozens of bodies orbiting it, all of which also have other orbiting
bodies.

During Earth operations on-orbit, the strongest gravitational
influences are the Earth, the Moon, the Sun, and Jupiter. The Earth
and the Moon are high on the list for obvious reasons, as is the
sun. Jupiter is also on the list because it's an exceptionally heavy
planet (over three times heavier than Saturn, the next heaviest
planet). Despite being so far away, it exerts a measurable influence
on satellites orbiting the Earth, and it must be taken into account
during the mission planning phase.

\illustration{perturbations}

When a spacecraft is affected by forces not taken into account in an
ideal two-body system, the orbital trajectory includes
perturbations. These can be caused by a non-spherical planet (such as
the earth), an uneven mass distribution of the planet, or other
planets. Perturbations manifest as slight drifts to the orbit over
time; the spacecraft may need to perform correction burns as a
result. The Earth itself isn't perfectly spherical; its gravitational
force varies depending on the spacecraft's inclination. Of other
planets, the Earth's moon is very large relative to the Earth for a
moon; its effects on orbiting spacecraft is marked, and special care
must be taken during mission planning for lunar perturbations.

Mathematical equations have been developed to quantify the influence
the moon has on Earth-based spacecraft orbits; together, they are
called lunar theory. Taking into account lunar theory is an important
part of planning Earth-based missions since the moon strongly affects
any satellite or spacecraft orbiting Earth due to its high mass and
relatively close proximity to the Earth.

\subsection{Eccentricity}

So far, we have only investigated perfectly circular orbits. In
reality, no orbit is perfectly circular; such an ideal orbit has an
eccentricity of zero. The more elliptical an orbit is, the higher its
eccentricity; an orbit with an eccentricity of one is a parabolic
escape trajectory; and an orbit with an eccentricity greater than one
is a hyperbolic trajectory.

\illustration{eccentricity}

When a body is orbiting around its parent with a parabolic escape
trajectory, its speed is the absolute minimum required to escape the
gravitational influence of its parent; if its speed were any lower,
the eccentricity would be less than one, and it would remain in orbit
instead of being flung out.

A hyperbolic trajectory is simply a parabolic escape trajectory, but
with a higher velocity. In reality, any spacecraft traveling between
two bodies will use a hyperbolic trajectory, both when departing the
first body and when arriving at the second.

\subsection{Inclination}

Unlike the orbits explained in these diagrams, reality is not
two-dimensional. Spacecraft must deal with a third dimension:
inclination.

When launching from an ideal equatorial launch site, the inclination
of the resulting orbit can be as low as zero degrees. The ground track
of a zero-inclination orbit would follow along the equator; if you
mounted a camera onboard the spacecraft and aimed it at the Earth, the
equator would always cross through the center of the frame.

\illustration{inclination}

However, no launch sites currently in operation are situated on the
equator. Cape Canaveral is at 28 degrees north of the equator, while
Russia's Baikonur Cosmodrome is nearly 46 degrees north of the
equator. When a spacecraft is launched into orbit, the minimum
inclination is the same as that of its launch site. This is a large
part of the decision to build the ISS at 51.6 degrees inclination,
since Russia's launch site is located at 46 degrees north; if the
inclination of the ISS was lower than 46 degrees, Russia would not be
able to easily launch their spacecraft to the ISS.

There's another factor that must be taken into consideration in
conjunction with orbital inclination, and that's the velocity of the
planet's surface at the launch site. Since the Earth spins along its
axis once every 23 hours and 56 minutes (one sidereal day), some
velocity can be taken from the Earth's rotation when launching
spacecraft. At the equator, the Earth's surface velocity (relative to
the Sun and Earth's center of mass) is 464 meters per second
(m/s). The further the launch site is from the equator, the less
surface velocity is present. To launch into orbit from a high
inclination launch site requires more energy from the launch vehicle
as compared to an equatorial launch site.

\illustration{latitude-velocity}

This is why most launch sites are located as close as possible to the
Earth's equator: the closer the launch site is to the equator, the
less fuel a launch vehicle will need to reach Earth orbit (and
beyond).

\subsection{Periapsis and Apoapsis}

The periapsis of an orbit is the point at which the spacecraft and
body are the closest to each other; the apoapsis is simply the
furthest point. Any orbit with an eccentricity greater than one will
have an apoapsis and a periapsis. The periapsis and apoapsis are
typically measured from the spacecraft to the surface of the body, to
simplify matters for the astronauts; however, mathematically, the
periapsis and apoapsis are measured relative to the barycenter.

\illustration{periapsis-apoapsis}

Since periapsis and apoapsis describe locations relative to the orbit,
not the parent body, any changes to the orbit will also change the
periapsis and apoapsis.

\subsubsection{Semi-major Axis}

The semi-major axis of an orbit is the distance from the periapsis to
the apoapsis, divided by two. Despite appearing to be derived from the
periapsis and apoapsis, the semi-major axis is the root from which
those are derived; since an orbit with an eccentricity less than one
is an ellipse with the parent body at one of the foci, the semi-major
axis is generally more useful for orbital calculations than the
periapsis and apoapsis are.

\illustration{semimajor-axis}

\subsubsection{Orbital Period}

The period of an orbit is the amount of time it takes to complete one
full orbit. It can be calculated from the semi-major axis and the mass
of the parent body. Due to Newton's laws, a higher orbit (with a
larger semi-major axis) will have a longer period than that of a lower
orbit. To adjust the orbital period, the semi-major axis must change;
this is done by changing the apoapsis and the periapsis by burning
prograde or retrograde.

\subsection{Ascending and Descending nodes}

Every object orbiting a body with an inclination greater than zero
will have both an ascending node and a descending node in its
orbit. Like the periapsis and apoapsis, the ascending and descending
nodes are located relative to the orbit and the parent body, so any
changes to the orbit will also change the ascending and descending
node.

Ascending and descending nodes only work if the parent body has an
equator. The imaginary plane extending out in all directions from the
equator of the parent body is called the plane of reference; it
applies to any satellites orbiting the body.

The ascending node is the point where the orbit crosses the plane of
reference, traveling south to north. The descending node simply
delineates the opposite side of the orbit, where it crosses the plane
of reference, travel-ling north to south.

\subsubsection{Argument of Periapsis}

The argument of periapsis is the angle, measured starting from the
ascending node and travel-ling northwards, to the periapsis. The
argument of periapsis orients the orbit's semi-major axis within its
orbital plane.

\subsubsection{Longitude of the Ascending Node}

The longitude of the ascending node is the longitude, relative to the
parent body's center of mass, is the longitude directly underneath the
ascending node. It is measured relative to the parent body's center of
mass, not its surface; this means that the longitude of the ascending
node will not change as the parent body spins on its axis.

The longitude of the ascending node is used to orient the orbit
relative to the parent body.

\subsection{Mean Anomaly at Epoch}

The mean anomaly at epoch describes the virtual angular position of
the satellite at a predetermined time (the epoch). The mean anomaly
does not describe a physical angle, except in the case of a perfectly
circular orbit; because of Kepler's third law, any orbit with an
eccentricity greater than zero will have a non-uniform orbital
velocity. Mean anomaly will therefore ``drift'' during the orbit; so
it must first be converted into true anomaly. True anomaly takes into
account the eccentricity of the orbit, and so provides the direct
physical angle between the periapsis and the orbiting object at a
given point in time.

\subsection{Keplerian Elements}

When the values for eccentricity, inclination, semi-major axis, the
argument of periapsis, the longitude of the ascending node, and mean
anomaly at epoch are combined, the exact position of the spacecraft in
space can be calculated at any time. Together, these six values are
called the orbit's Keplerian elements. They only apply to an ideal
two-body system, with no external gravitational influences. Since
external gravitational influences affect the orbit over time,
Keplerian elements cannot be accurately used for anything beyond an
ideal two-body system.

\section{Orbital Maneuvering}

Maneuvering in orbit is a critical component of any spacecraft
operation, and is required for orbital rendezvous and docking as well
as deorbit operations. It is also essential during any mission that
flies beyond low earth orbit.

Orbital mechanics make maneuvering difficult and counter-intuitive.

All orbital maneuvers simply adjust your orbit, albeit in different
directions. There are a few main categories of orbital maneuvering:

\vspace{2em}

\begin{enumerate}
\item Periapsis and apoapsis adjustments (changing your altitude)
\item Orbit phasing (adjusting the duration, or period, of your orbit)
\item Plane changes (also known as inclination changes)
\end{enumerate}

\subsection{Impulsive vs. Finite Maneuvers}

Any maneuver can be ideally simulated as an impulsive maneuver, a
maneuver which is completed instantly. In reality, an impulsive
maneuver is impossible as it would require infinite thrust; however,
it's a good approximation during mission planning. The analogous
non-instant maneuver is called a finite maneuver, as it takes a finite
amount of time to complete.

There are methods to convert an impulsive maneuver into a nearly
equivalent finite maneuver; thus, most of the mission planning can be
accomplished with impulse maneuvers, to be converted into finite
maneuvers after verifying the intermediate orbits.

The following sections will assume impulse maneuvers for
simplicity. In reality, these maneuvers will take a finite amount of
time, so the start of the maneuver must be shifted backwards in time
so that the midpoint of the maneuver is at the proper location.

\subsection{Orientation}

As you might guess, orientation is critical during orbital maneuvers,
because most spacecraft have an optimum direction to thrust in. Some
spacecraft have engines that are specifically designed for orbital
maneuvers, but they only point in one direction. For these reasons,
spacecraft need to orient themselves while performing orbital
maneuvers.

Most maneuvers involve orienting the ship either prograde (in the
direction of movement) or retrograde (the exact opposite of
prograde). Inclination changes, however, require orientations that
point transversely relative to the orbital plane (i.e. ``north'' or
``south'', but relative to the orbital plane). These are called normal
(for north-facing) and anti-normal (for south-facing).

Very rarely used are radial orientations; a radial-in orientation is
perpendicular to the orbital trajectory, pointing in the general
direction of the planet; a radial-out orientation is the opposite.

\subsection{Periapsis and Apoapsis Adjustments}

To adjust the periapsis or apoapsis, the spacecraft must be at the
opposite point in the orbit. For example, to increase the apoapsis,
the spacecraft must be at the periapsis, then burn prograde until the
apoapsis reaches the target value. The reverse is true to lower the
apoapsis.

\illustration{raise-apoapsis}

Since the orbital period depends on the length of the semi-major axis,
with a larger orbit having a longer orbital period, adjusting the
periapsis and apoapsis will by necessity also adjust the orbital
period.

Often, a spacecraft needs to adjust both its periapsis and apoapsis by
the same amount. Usually, the most efficient method is a Hohmann
transfer. To perform a Hohmann transfer, the spacecraft first performs
an orbit adjustment maneuver to change the altitude of the other side
of its orbit; then, when it reaches the new altitude at the other side
of its orbit, it performs another orbit adjustment burn to bring the
first point to the proper altitude as well.

\subsection{Orbit Phasing}

Orbit phasing refers to the timing of a spacecraft within its
orbit. This is usually required during rendezvous; even if the
semi-major axes, orbital planes, and eccentricities of the two
spacecraft are the same, the spacecraft may be a half-orbit
apart.

Orbit phasing needs to be performed on every mission that includes a
rendezvous. However, with the proper planning, it can be performed
alongside the spacecraft's altitude adjustment maneuvers.

\subsection{Plane Changes}

A plane change maneuver adjusts the inclination of the spacecraft's
orbit. To do so, it must burn in the normal or anti-normal
directions. Plane change maneuvers are the most expensive maneuver, \dv\ 
wise. To adjust the inclination of a circular orbit by 90 degrees
requires nearly as much \dv\ as it took to reach orbit in the first
place. Plane changes are only performed when they are absolutely
necessary, such as for satellites that must be in geostationary
equatorial orbit (GEO). Such satellites must be on the equator, and
since no launch site is on the equator, they must perform a plane
change once in orbit.

Since all orbital maneuvering \dv\ is relative to your existing \dv\ ,
plane changes are more efficient the more eccentric your orbit is,
provided the plane change maneuver is performed at the apoapsis. It's
often more efficient to dramatically increase the apoapsis, perform a
plane change, then lower the apoapsis again instead of performing a
plane change in-place. Some launch providers will launch GEO
satellites into a super-synchronous orbit, with an apoapsis much
higher than necessary, to reduce the \dv\ requirements of the satellite
itself.

\subsection{Oberth Effect}

The Oberth effect describes the counterintuitive fact that a prograde
burn at periapsis imparts more energy to the spacecraft than one
performed at the apoapsis. This is because the kinetic energy in an
object is the square of its velocity; therefore, adding a fixed amount
of velocity while the spacecraft is moving rapidly will add more
kinetic energy than if the velocity was added while the spacecraft was
moving slowly.

\section{Orbital Rendezvous}

One of the most difficult tasks is rendezvous and docking with another
spacecraft. It requires a deep understanding of orbital maneuvering,
the capabilities of your spacecraft, and the ability to precisely
target another spacecraft in orbit.

To understand orbital rendezvous, it's simpler to first work backwards
from a spacecraft that's already docked to a space station. Their
orbits are completely identical, and their speeds are matched up
perfectly.

First, the spacecraft must undock from the space station. This is
usually performed by sealing the hatch from both sides, unlocking the
docking port, then using electromechanical pushers to mechanically
separate the two. After some coast time, the spacecraft is far enough
away from the space station to use its RCS thrusters.

At this point, the spacecraft will begin a series of Hohmann
transfers, reducing its altitude (and increasing its distance to the
space station at the same time). This is done to minimize RCS thruster
usage while the spacecraft is still near the space station. From an
observer on the space station, the spacecraft will appear to fall
towards the planet and start moving retrograde.

Now, the spacecraft is in a separate orbit, lower and faster than that
of the space station. The spacecraft has just performed the opposite
of a rendezvous; the only difference is that a departure is typically
faster than a rendezvous, as there's less chance of approaching the
space station too closely.

The reverse of spacecraft departure is spacecraft rendezvous. Now, the
spacecraft is in the exact same low, fast orbit that is needed to
begin orbital rendezvous; with the space station in a higher, slower
orbit. The orbital planes are parallel; if not, a plane change
maneuver will have to be made.

The spacecraft now performs a series of Hohmann transfers, to raise
its orbit to that of the space station. The timing of these transfers
is critical, since the orbital phasing is being adjusted at the same
time. During the maneuvering, the spacecraft is communicating with
both the space station and ground controllers, who watch the position
to make sure the spacecraft will not pass too close to the space
station.

As the spacecraft approaches the space station from below, it performs
several small burns to match orbits nearly perfectly, then maneuvers
itself around the space station to a docking port; despite the close
proximity of the spacecraft and the space station, orbital mechanics
still applies, and care must be taken that the spacecraft remains in
the same position relative to the space station.

The only thing that remains is to dock with the space station; this is
usually fully automated, like rendezvous; during the docking
procedure, the spacecraft will use its onboard docking camera to track
visual targets on the space station's docking port.

When the spacecraft contacts the docking port, petal-shaped flaps
align the two docking ports; hooks extend to lock the two together,
and connectors for air and electricity are connected and opened. Soon
afterwards, the hatches on both sides are opened and crew members may
board the space station.

