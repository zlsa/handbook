
\chapter{Mission Operations}

\afterchapter

In reality, a crewed mission is much more complex than the orbital
maneuvers described above. In fact, with a mission that may only last
a few days, mission planners on the ground have been preparing the
mission for several years.

There are many routine operations that must be done during a
mission. These can be roughly be divided into four sections:

\begin{itemize}
\item Prelaunch operations
\item Launch operations
\item Postlaunch operations
\item Earth return operations
\end{itemize}

\section{Launch Operations}

\subsection{Prelaunch Operations}

Months before the launch of your spacecraft, mission planners will
begin to plan out your mission. If your mission is to a space station
such as the International Space Station, the mission planners will
need to secure approval for a visiting spacecraft as well as preparing
for the rendezvous procedure. Your mission planners will formulate a
mission plan. This includes:

\begin{itemize}
\item Cargo manifest and center of mass calculations
\item Astronaut scheduling
\item Launch profile and orbital trajectory
\item Space station approach, rendezvous, and docking
\item Space station undocking and departure
\item Deorbit planning and entry trajectory
\item Recovery vessel positioning
\end{itemize}

While your mission is being prepared by the mission planners, you and
your crew will train for spacecraft operations and zero-g movement, in
addition to whatever specialized training your mission may
require. Crew training is considered one of the more strenuous and
difficult parts of being an astronaut, but just remember: what you
learn in crew training will possibly save your life in the future.

\subsection{Launch Operations}

The spacecraft will arrive at the launch site integration hangar a few
weeks before launch. During this time, it is loaded with cargo, with
the exception of late-load cargo. Checks are done on the spacecraft,
and it's integrated to the launch vehicle and readied for rollout and
erection.

The launch vehicle is usually rolled out to the launchpad a few days
in advance of T-0. Different launch providers have different launch
vehicle designs, and hence have different rollout and erection
schedules and guidelines.

You and your crew will board the spacecraft only a few hours before
launch, before fueling has occurred. Typically, you will board the
spacecraft via the crew arm, which is swung away after ground crew
have strapped you into your seats and closed the spacecraft ground
door.

After the launchpad is cleared of all personnel and the command is
given, launch vehicle fueling will begin. Once again, precise timing
depends on the launch provider's choice of vehicle.

If your mission includes rendezvous with a space station, the launch
window is very short or instantaneous. If the launch occurs too far
from the optimal launch time, the spacecraft will need to perform an
expensive on-orbit inclination change. Most spacecraft do not have
enough onboard \dv\ to perform this maneuver themselves.

Fueling is usually completed a few minutes to an hour before the
planned liftoff time. The crew access arm retracts, and the launch
vehicle and spacecraft switch to internal power (instead of power
provided from the launch pad) in preparation for flight.

A few minutes before liftoff, the fueling valves are closed, and any
excess gases in the tanks are vented until seconds before engine
ignition. This is only an issue with cryogenic propellants, such as
liquid oxygen or liquid methane; kerosene, with a boiling temperature
well above the boiling point of water, does not boil off while in the
tanks.

A few seconds before liftoff, the engine ignition command is
given. Engine startup is a long and complicated process, and
auto-aborts during the ignition sequence are not rare. After the
vehicle has determined that all of the engines are operating properly
and at the right thrust levels, the command is given to release the
hold-downs.

As soon as the hold-downs are released, the launch vehicle will begin
to ascend. A few seconds after liftoff, the launch vehicle will have
cleared the tower; soon afterwards, the vehicle will perform a pitch
kick to begin the gravity turn.

As the propellants are burned in the engines to produce thrust, the
launch vehicle will lose mass; however, since the thrust level of the
engines does not decrease, the thrust-to-weight ratio (TWR) will
increase during first-stage flight. To compensate for increased
acceleration, launch vehicles need to reduce thrust. There are two
primary ways to reduce thrust; the method depends on the
throttleability of the launch vehicle's chosen rocket engine.

On launch vehicles whose engines cannot throttle (or if the engines do
not have the necessary throttle range), one or more engines are shut
down to reduce total thrust. This method only works if you have three
or more engines, and if they are arranged to allow for shutdowns
without adversely affecting the thrust vector.

The other, more common method is to throttle all of the engines down
equally. This is the preferred method for many reasons: first,
throttling engines is much smoother than shutting them down; second,
engine shutdown is nearly as complex as engine ignition; and third,
shutting down engines changes the thrust distribution. All
currently-flying human-rated launch vehicles throttle their engines
during the ascent.

During the atmospheric portion of ascent, the launch vehicle must fly
in the direction it's traveling; even slight deviations will cause a
rapid loss of control due to aerodynamic forces and necessitate a
launch abort.

As the vehicle ascends, there will be a point of maximum dynamic
pressure (MaxQ) on the vehicle; this is the point of highest
aerodynamic stress on the vehicle and is one of critical points during
a launch. In most launch vehicles, MaxQ occurs shortly after the
launch vehicle goes supersonic.

When the target velocity and altitude are reached, the first stage
engines shut down; this is called main engine cutoff (MECO). Shortly
afterwards, the first stage is separated and the second stage engine
is ignited.

Depending on how many stages the launch vehicle has, this cycle
continues until the spacecraft has reached its desired orbit (or in
the case of a mission to a space station, a preliminary parking
orbit). At this point, the engine on the final stage is shut down and
the spacecraft is separated from the launch vehicle.

\subsection{Postlaunch Operations}

The primary concern after reaching orbit is power. Most spacecraft
feature unfolding solar arrays, but some spacecraft, such as the
SpaceX Crew Dragon, feature integrated solar panels that do not
require any action to provide power. On the other hand, the Boeing
Starliner does not feature any type of power generation; as such, the
Starliner is only capable of visiting space stations that can provide
power to visiting vehicles.

Once the spacecraft is in the correct attitude and is generating
power, less important systems can be checked for proper functionality.

\subsection{Earth Return Operations}

After the mission is complete, the spacecraft and its crew must return
to Earth. The service module is usually jettisoned after the deorbit
burn; it burns up in the atmosphere and is lost.

For more information, \seeref{ch:planetary-landing}.
