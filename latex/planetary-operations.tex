
\chapter{Planetary Operations}

\section{Departing from a planet}

To depart from a planet, the spacecraft must be in an orbit with an
eccentricity greater or equal to one; i.e. a hyperbolic orbit. The
maneuver used to raise the apogee beyond the planet's sphere of
influence is called an escape burn.

Escape burns are more efficient at the periapsis of the spacecraft's
orbit due to the Oberth effect; to perform an escape burn, the
spacecraft simply raises its apogee until the eccentricity of its
orbit is greater than or equal to one. Escape burns must be precisely
plotted ahead-of-time to ensure the escape trajectory intersects the
destination planet at the right time, angle, and speed.

\section{Traveling to a Planet}

The travel time of a spacecraft depends on the position of the planets
and on the trajectory set by the initial departure burn. Traveling to
and from Mars takes about 6-9 months, depending on how energetic the
trajectory is; shorter transfer times necessitate a higher-energy
departure burn.

When not near the Earth, the spacecraft will not have the protection
of a magnetic field; manned missions venturing beyond LEO need
radiation protection.

\section{Arriving at a Planet}

When arriving at a planet, the spacecraft will be on a hyperbolic
trajectory. With no further corrections, the spacecraft will either
swing around the planet and begin to fly away from the planet (a
``flyby'') or enter the atmosphere (if present) and crash into the
surface at high speeds.

A flyby might sound like a poor plan for any mission; however, it's
commonly used on unmanned missions that have a very limited mass
budget. Since a flyby uses no fuel (apart from correction burns), the
spacecraft can be much lighter, providing more extra mass for
scientific instruments and other additions.

Most manned spacecraft must either descend to the surface of the
planet or enter an orbit around the planet. The former requires less
fuel, but since the spacecraft cannot delay landing in the case of a
hardware fault, it's also riskier.

\subsection{Entering Orbit}

To enter into an orbit around a planet, the spacecraft must perform a
capture maneuver which slows the vehicle down at its periapsis. To do
so, it must use some combination of a retrograde engine burn and
aerobraking.

\subsubsection{Retrograde Engine Burn}

A retrograde engine burn capture is performed by orienting the
spacecraft retrograde, then performing a long engine burn at periapsis
to reduce the spacecraft's apoapsis. This method does not require a
heatshield and works on planets without atmospheres.

\subsubsection{Aerobraking}

Aerobraking is a maneuver which uses the atmosphere of a planet to
lower the spacecraft's apogee without landing. The spacecraft must
have a shield of some sort to dissipate the heat generated during
atmospheric flight, and the planet must have an atmosphere to brake
against.

This method is far superior to an engine burn in terms of mass
penalty, but it imposes strict limits on the spacecraft's design, as
it must withstand very high G-forces. In addition, the spacecraft
trajectory must be accurately controlled to ensure a precise orbit
after the maneuver. This is because atmospheres vary over time, and
the overall braking force can't be calculated accurately beforehand.

\section{Landing on a Planet}

To land on a planet, there are three distinct steps that must be
performed (assuming the spacecraft is currently in low orbit around
the planet).
%
First, the spacecraft must lower its periapsis to an altitude that's
appropriate for the chosen method of braking (either with a
\linkref{subsec:deorbiting}{deorbit burn} or from an interplanetary
hyperbolic trajectory);
%
then, the spacecraft must reduce its orbital velocity through
\linkref{subsubsec:atmospheric-entry}{atmospheric entry} or with a
\linkref{subsubsec:braking-burn}{braking burn};
%
and finally, the spacecraft must slow itself down to speeds
appropriate for a soft landing.

\subsection{Deorbiting}
\label{subsec:deorbiting}

Before a spacecraft can land, its orbit must pass close to the
surface. In the case of a planet with an atmosphere, the orbit does
not need to intersect the surface, since the atmosphere will naturally
slow down the spacecraft as it descends towards the surface. On
planets without atmospheres, such as the moon, the deorbit burn is
split into two parts; first, the spacecraft performs a
periapsis-lowering burn to bring the periapsis closer to the surface;
the second part of the deorbit burn, the braking burn, is used to
reduce the spacecraft's orbital speed to zero and land on the planet's
surface.

Deorbiting a spacecraft is relatively simple: the spacecraft must
perform a retrograde burn maneuver at the apoapsis that lowers the
periapsis. If an atmosphere is present, the deorbit burn also
fine-tunes the entry trajectory.

\subsubsection{Atmospheric Entry}
\label{subsubsec:atmospheric-entry}

Atmospheric entry is essentially an aerobraking pass that reduces the
spacecraft's velocity enough to bring the periapsis and apoapsis into
the atmosphere.

\illustration{reentry-capsule-control}

When the spacecraft reaches the atmospheric interface, it will engage
active entry guidance. Every modern crewed spacecraft is capable of
guided entry; even capsule-derived spacecraft can control their
trajectory with an offset center of mass and precisely controlled roll
around the vertical axis. Most spaceplanes perform wide S-turns to
decrease peak G-forces.

% S-turns illustration

\illustrationsoon

During the time of peak heating and G-forces on the spacecraft,
communications will cut out as a consequence of the superheated plasma
surrounding the spacecraft; communications blackout typically lasts
only a few minutes. During this time, the spacecraft experiences high
G-forces, typically in the range of 3-5 Gs (that is, 3-5 times more
force than on Earth); spaceplanes usually experience less G-force as a
virtue of their increased lift-to-drag ratio. Every crewed spacecraft
is designed for these forces, and furthermore, they're designed to
safely and comfortably secure the astronauts during this time.

\subsubsection{Braking Burn}
\label{subsubsec:braking-burn}
  
Atmospheric entry is only possible on planets with atmospheres;
planets such as the Moon don't have any atmosphere to slow down
incoming spacecraft, so aerobraking isn't possible on the Moon. For
landing on planets without atmospheres, a long braking burn is
necessary to bleed off the spacecraft's orbital velocity.

% ??? Insert braking burn image here

\illustrationsoon

Braking burns can also be used during atmospheric entry, for
additional drag and braking force; this is known as
``retropropulsion''.

While Mars has a very thin atmosphere---less than 1\% of
Earth's---it's still enough to slow down incoming spacecraft to a
fraction of their orbital velocity. However, parachutes by themselves
can't be used on Mars for landing; since aerodynamic drag is based on
the square of the velocity, a parachute with twice the area won't slow
the vehicle down to half the speed. Spacecraft landings on Mars are
typically parachute-assisted or fully propulsive.

\section{Planetary Landing}
\label{ch:planetary-landing}

\afterchapter

Successfully landing on a planet requires a set of steps to be
performed with precise timing. The steps required for a landing change
if the planet has no atmosphere; for example, aerobraking and
parachutes are rendered useless, so all braking must be performed
propulsively.

\section{Landing}

\subsection{Parachute Splashdown/Airbag Assist}

Most modern capsule-derived spacecraft use parachutes for a splashdown
in the ocean or an airbag or rocket-assisted land landing. This method
is reliable and proven, and is relatively simple compared to other
choices. This method deploys drogue chutes after the capsule
decelerates to subsonic velocities, then uses the drogue chutes to
pull out the main chutes, of which there are usually three or
four. After a few minutes of slow descent under the parachute canopy,
the capsule either splashes down in the ocean or deploys airbags or
rocket motors, to soften a land-based touchdown.

\illustrationsoon

Parachute landings are the tried-and-true technique, but they suffer
from difficult recoveries (in the case of ocean splashdowns) and hard
landings, even when airbag or rocket-assisted (in the case of land
landings). Ocean splashdowns also expose the spacecraft to salty ocean
water only minutes after the spacecraft has been exposed to the
extreme temperatures of atmospheric entry.

\subsection{Propulsive Landing}

Some next-generation spacecraft feature propulsive landing. Instead of
using a parachute to reduce their speed, they use multiple onboard
rocket motors to slow the vehicle down in a precise,
computer-controlled manner. The primary advantage of propulsive
landing is the increased accuracy and no chance of tangled parachutes;
on the other hand, propulsive landing is a relatively new addition to
crewed capsules, and reliability hasn't been proven yet. In addition,
propulsive landing does not depend on an atmosphere; propulsive
landing works just as well on Mars as it does on Earth.

\illustrationsoon

Propulsively-landed spacecraft cannot be landed manually; humans
simply cannot control the spacecraft quickly or accurately enough for
a manual landing. Spacecraft with propulsive landing will need to have
a parachute backup in the near future until the reliability of
propulsive landing is proven.

\subsection{Spaceplane Runway Landing}

Currently, there are no crewed spaceplane designs; but in the future,
Sierra Nevada Corporation's {Dream Chaser} vehicle might change
this. Nevertheless, we will provide a description of what to expect
when landing a spaceplane on a runway.

Unlike the far simpler parachute splashdowns in the ocean, a runway
landing requires highly accurate computerized entry guidance, to
ensure the spacecraft is able to reach the target runway. Furthermore,
no spaceplane currently in service features atmospheric engines;
therefore, the flight must be flown unpowered.

As airplanes cannot gain altitude without losing speed, all spaceplane
entry trajectories must place the spaceplane on a trajectory that
overflies the runway; after reentry, as the spaceplane slows down from
hypersonic speeds, it will perform wide S-turns to bleed off speed and
altitude without gaining as much ground as a direct glide would.

When the spaceplane reaches the appropriate altitude and is aligned to
the runway, the spaceplane will autonomously descend along the
glideslope, deploy its landing gear, and rollout along the runway
automatically. Modern spaceplanes are difficult to control manually,
so the pilot will only take over in the case of an anomaly with the
autopilot.
