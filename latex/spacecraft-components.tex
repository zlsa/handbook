
\chapter{Spacecraft Components}

\section{Service Module}

Most crewed spacecraft feature a service module. Much of the equipment
used in space does not need to be safely returned to Earth, such as
solar panels and orbital maneuvering engines. These are usually built
into the service module instead; after the deorbit burn, the service
module is jettisoned before reentry.

\illustration{service-module}

\section{Pressure Vessel}

Every crewed spacecraft needs a pressurized compartment for its
crew. The crew compartment must be pressurized to withstand the vacuum
of space, and it must provide environmental controls and life support
systems (\acrshort{eclss}) for the long stay in space. The crew compartment
usually contains custom-made seats, for the astronauts to sit in
during launch and reentry, in addition to on-orbit sleeping when not
docked to a space station. The crew compartment is not necessarily the
same as the spacecraft's outer mold line (the ``shape'' of the vehicle).

\illustration{pressure-vessel}

\subsection{ECLSS}

Environmental controls and life support systems (\acrshort{eclss}) are
a required component of any crewed vehicle. Without \acrshort{eclss}, exhaled
carbon dioxide will displace the oxygen, and the crew compartment will
overheat when the capsule is exposed to the sun. \acrshort{eclss} is a very
complicated mission-critical system, and redundancy is built-in so
that an \acrshort{eclss} subcomponent failure does not necessitate a mission
cancellation.

\section{Power}

Every spacecraft needs power; some spacecraft use fuel cells and some
use batteries. Most battery-powered spacecraft also feature some form
of recharging the batteries while the spacecraft is in space; these
usually take the form of solar panel arrays.

Power is a critical system on any spacecraft since it powers many
other critical systems, such as \acrshort{eclss}. Since astronauts cannot simply
perform an \acrshort{eva} to debug the power system, it must be incredibly
reliable and fault-tolerant.

\subsection{Power Generation}

Not all spacecraft have power generation; for example, the Boeing
Starliner only has batteries, and is not capable of free-flight for
more than 60 hours; it must rendezvous and dock with a space station
for any mission that's longer than 60 hours. Any spacecraft capable of
deep-space flight needs some form of power generation.

\subsubsection{Solar Arrays}

Solar arrays are used to recharge the on-board batteries. Solar arrays
are almost always installed on the service module, as they're not
needed during reentry and landing; placing them on the service module
reduces the vehicle's mass during reentry.

Usually, the solar arrays unfold once the spacecraft has reached
orbit; however, some spacecraft feature fixed solar panels that are
mounted to the service module. Fixed panels, unlike deployable solar
arrays, cannot fail to deploy and usually weigh less; however, the
maximum amount of power they generate is limited by the size of the
service module.

\subsection{Power Storage}

Power storage is required; during the launch, solar arrays cannot
point toward the sun or even be extended, and for long periods of time
while in orbit, the sun will be blocked by the Earth itself. For this
reason, all spacecraft need some way to store power for long periods
of time.

\subsubsection{Fuel Cells}

Fuel cells combine hydrogen and oxygen to produce water; as a
byproduct, electricity is generated. In the case of a hydrogen-oxygen
fuel cell, the chemical reaction can be reversed with large amounts of
energy to convert water into hydrogen and oxygen.

In the past, fuel cells had a higher energy density than batteries;
nowadays, lithium-based batteries are denser than fuel cells. Fuel
cells don't degrade as much over time as batteries do. However, fuel
cells aren't as efficient as batteries; losses are usually over 50\%.

Since fuel cells produce water as a waste product, fuel cells can be
used to provide drinkable water to the crew as well.

\subsubsection{Batteries}

Batteries store energy chemically; unlike fuel cells, there's no
liquid fuel to speak of. Lithium-based batteries are the clear winner
so far in terms of energy density per kilogram; however, lithium-based
batteries need to be climate-controlled and carefully charged to avoid
thermal runaway and catastrophic battery failure.

Not all lithium-based batteries are rechargeable; non-rechargeable
batteries usually have higher energy densities than rechargeable
variants.

\section{Control Systems}

All spacecraft need to maneuver once in orbit, if only to deorbit at
the end of the mission. Most spacecraft also need to keep themselves
pointing in one direction to keep the solar panels facing towards the
sun and to keep the thermal radiator facing away from the sun.

\subsection{Thrusters}

There are a few methods of controlling attitude on a spacecraft, but
by far the most common is a reaction control system (RCS). A reaction
control system is simply a network of small thrusters, that can
produce small amounts of thrust quickly and accurately. With enough
RCS thrusters pointed in different directions, the spacecraft can
orient itself in any direction.

RCS thrusters are usually placed directly in the capsule or
spaceplane, but some spacecraft feature RCS thrusters on the service
module in addition to the capsule or spaceplane RCS thrusters. RCS
propellant is sometimes stored in the service module as well; however,
fuel transfer pipes are then required, which adds complexity to the
spacecraft/service module interface and adds risk to the service
module jettison sequence.

\subsection{Engines}

Some spacecraft also feature a dedicated orbital engine, usually
pointed directly backwards. This engine is usually more powerful and
efficient than RCS thrusters; these engines are used when performing
long orbital maneuvering burns, such as the deorbit burn.

Space stations tend to be delicate; typically, visiting spacecraft
must not use their RCS thrusters when in close proximity to the
station. This is to prevent damage of delicate space station
components, such as solar arrays and wiring.

